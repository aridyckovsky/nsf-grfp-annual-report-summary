\documentclass{article}
\usepackage[margin=1in, headsep=0]{geometry}

\title{Fellowship Year Summary}
\date{2023--24}
\author{Ari M. Dyckovsky}

\begin{document}

\maketitle
\thispagestyle{empty}

\textbf{Intellectual Merit.} My research investigates how communities of people value things. Over the past year, I developed a psychologically-grounded theoretical framework to explain how communities construct the value of objects (e.g., mugs, homes, stocks) over time. I'm preparing a manuscript detailing the framework and its implications for submission to an interdisciplinary peer-reviewed journal this summer. This framework motivates two ongoing empirical projects---one observational \& one experimental---to examine how interactions shape, and are shaped by, our perceptions of value. To develop these empirical directions, I applied for and was awarded a grant by Princeton University's Data-Driven Social Science Initiative. For the observational project, I designed a big data study that explores the social construction of value in cryptocurrency token markets (e.g., Bitcoin). I've developed reproducible cloud-based infrastructure that programmatically collects and prepares historical data from social media platforms (Twitter, Reddit, and Telegram) and cryptocurrency exchanges (Coinbase, Binance, Kraken, etc.). I'm extending this infrastructure to analyze the dynamic relationship between social media conversations and cryptocurrency trading decisions before and after shocks (e.g., the collapse of FTX in late 2022). I expect to find that cryptocurrency token communities exhibiting a stronger vs. weaker sense of shared identity on social media tend to be more vs. less resilient to shocks, i.e., stronger community identities relate to higher token valuations. For the experimental project, I'm developing reusable web-based infrastructure to create and run interactive network experiments. This infrastructure will be used to test the extent to which a sense of shared identity moderates individual- and collective-level perceptions of value. This and other similar experiments will use an open-source tool I've created to generate social network structures and randomly assign participants in them, enabling both reproducible and controlled investigations of interacting communities over time. 

\textbf{Broader Impacts.} My lab will use the infrastructures and tools I'm developing to pursue a variety of psychological research directions that involve conversations in networks. I'm building both empirical systems to be open-sourced for other investigators to use in their own investigations of critical issues affecting society, from political polarization to the misinformation epidemic. For instance, social scientists may configure my web-based experiment infrastructure to test interventions for climate change beliefs using a framework for understanding belief dynamics proposed in a paper I published this year with Drs. Madalina Vlasceanu and Alin Coman. I continue to make improvements to the broader research community through service programs and open source contributions. In my fourth year as Director of Core Systems of the Application Statement Feedback Program (ASFP), the systems I designed and maintain enabled 562 applicants to psychology PhD programs to receive 1,200 pieces of expert feedback on their statements of purpose from 235 trained editors across the U.S. (PhD students, postdocs, and faculty). The ASFP team received the 2023 Mission Award from the Society for the Improvement of Psychological Science for our work to make psychological science more diverse and inclusive. I've contributed to various open-source projects, including \textit{Empirica}, that focus on making research tools globally accessible. I'm excited to continue building inclusive, open-access programs and tools that make science better.

\end{document}
